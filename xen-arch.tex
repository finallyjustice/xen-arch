%%%%%%%%%%%%%%%%%%%%%%%%%%%%%%%%%%%%%%%%%
% Beamer Presentation
% LaTeX Template
% Version 1.0 (10/11/12)
%
% This template has been downloaded from:
% http://www.LaTeXTemplates.com
%
% License:
% CC BY-NC-SA 3.0 (http://creativecommons.org/licenses/by-nc-sa/3.0/)
%
%%%%%%%%%%%%%%%%%%%%%%%%%%%%%%%%%%%%%%%%%

%----------------------------------------------------------------------------------------
%	PACKAGES AND THEMES
%----------------------------------------------------------------------------------------

\documentclass[aspectratio=169]{beamer}
%\documentclass{beamer}

\mode<presentation> {

% The Beamer class comes with a number of default slide themes
% which change the colors and layouts of slides. Below this is a list
% of all the themes, uncomment each in turn to see what they look like.

%\usetheme{default}
%\usetheme{AnnArbor}
%\usetheme{Antibes}
%\usetheme{Bergen}
%\usetheme{Berkeley}
%\usetheme{Berlin}
%\usetheme{Boadilla}
%\usetheme{CambridgeUS}
%\usetheme{Copenhagen}
%\usetheme{Darmstadt}
%\usetheme{Dresden}
%\usetheme{Frankfurt}
%\usetheme{Goettingen}
%\usetheme{Hannover}
%\usetheme{Ilmenau}
%\usetheme{JuanLesPins}
%\usetheme{Luebeck}
\usetheme{Madrid}
%\usetheme{Malmoe}
%\usetheme{Marburg}
%\usetheme{Montpellier}
%\usetheme{PaloAlto}
%\usetheme{Pittsburgh}
%\usetheme{Rochester}
%\usetheme{Singapore}
%\usetheme{Szeged}
%\usetheme{Warsaw}

% As well as themes, the Beamer class has a number of color themes
% for any slide theme. Uncomment each of these in turn to see how it
% changes the colors of your current slide theme.

%\usecolortheme{albatross}
%\usecolortheme{beaver}
%\usecolortheme{beetle}
%\usecolortheme{crane}
%\usecolortheme{dolphin}
%\usecolortheme{dove}
%\usecolortheme{fly}
%\usecolortheme{lily}
%\usecolortheme{orchid}
%\usecolortheme{rose}
%\usecolortheme{seagull}
%\usecolortheme{seahorse}
%\usecolortheme{whale}
%\usecolortheme{wolverine}

%\setbeamertemplate{footline} % To remove the footer line in all slides uncomment this line
%\setbeamertemplate{footline}[page number] % To replace the footer line in all slides with a simple slide count uncomment this line

%\setbeamertemplate{navigation symbols}{} % To remove the navigation symbols from the bottom of all slides uncomment this line
}

\usepackage{graphicx} % Allows including images
\usepackage{booktabs} % Allows the use of \toprule, \midrule and \bottomrule in tables
%\RequirePackage{beamerthemeoracle}

%----------------------------------------------------------------------------------------
%	TITLE PAGE
%----------------------------------------------------------------------------------------

\title[Xen is not just paravirtualization]{Xen is not just paravirtualization} % The short title appears at the bottom of every slide, the full title is only on the title page

\author{Dongli Zhang} % Your name
\institute[Oracle] % Your institution as it will appear on the bottom of every slide, may be shorthand to save space
{
Oracle Asia Research and Development Centers (Beijing) \\ % Your institution for the title page
\medskip
\textit{dongli.zhang@oracle.com} % Your email address
}
\date{\today} % Date, can be changed to a custom date

\begin{document}

%------------------------------------------------

\section{Title}
\begin{frame}
\titlepage % Print the title page as the first slide
\end{frame}

%------------------------------------------------

\section{Plan}
\begin{frame}
\frametitle{Plan}
\begin{itemize}
\setlength\itemsep{1em}
\item {\large Virtualization}
\item {\large Xen Virtualization}
\end{itemize}
\pause
\begin{figure}
\includegraphics[width=1.0\linewidth]{figures/plan.pdf}
\end{figure}
\end{frame}

%------------------------------------------------

\section{What is virtualization}
\begin{frame}
\frametitle{What is virtualization}
\begin{itemize}
\item A virtual machine is taken to be an efficient, isolated duplicate of the real machine
(by Formal Requirements for Virtualizable Third Generation Architectures, Gerald J.Popek and Rebert P. Goldberg, 1974) \pause
\end{itemize}
\begin{center}
\begin{figure}
\includegraphics[width=0.3\linewidth]{figures/sparc.pdf}
\includegraphics[width=0.2\linewidth]{figures/power.pdf}
\end{figure}
\end{center}
\end{frame}

%------------------------------------------------

\section{Trap and Emulate}
\begin{frame}
\frametitle{Trap and Emulate}
\begin{itemize}
\item Virtual Machine (Guest) at \textbf{Unprivileged Mode}
\item Virtual Machine Monitor (Host or Hypervisor) at \textbf{Priviledged Mode}
\end{itemize}
\begin{figure}
\includegraphics[width=0.8\linewidth]{figures/trapemu.pdf}
\end{figure}
\end{frame}

%------------------------------------------------

\section{x86 is NOT virtualizable}
\begin{frame}
\frametitle{x86 is NOT virtualizable}
\begin{itemize}
\item Virtualizable Architecture: all \textbf{sensitive instructions} must also be \textbf{privileged instructions} (by Gerald J.Popek and Rebert P)
\item \textbf{\textcolor{red}{critical instructions}} = $sensitive \ instructions - privileged \ instructions$ \pause
\item 18 critical instructions on x86 (Analysis of the Intel Pentium's Ability to Support a Secure Virtual Machine Monitor. USENIX Security 2000):
	\begin{itemize}
		\item \textit{SGDT/SIDT/SLDT, SMSW, PUSHF/POPF}
		\item \textit{LAR/LSL, VERR/VERW, POP/PUSH}
		\item \textit{CALL, JMP, INT n, RET}
		\item \textit{STR, MOV}
	\end{itemize}
\pause
\item Solutions:
	\begin{itemize}
		\item Binary Translation (QEMU, VMWare)
		\item Paravirtualization (Xen)
		\item Hardware-Assisted Virtualization (Xen, KVM, VMWare based on Intel-VT and AMD-V)
	\end{itemize}
\end{itemize}
\end{frame}

%------------------------------------------------

\section{Solution 1/3: Binary Translation}
\begin{frame}
\frametitle{Solution 1/3: Binary Translation}
\begin{itemize}
\item \textbf{philosophy: rewrite critical instructions}
\end{itemize}
\begin{figure}
\includegraphics[width=1.0\linewidth]{figures/qemu.pdf}
\end{figure}
\end{frame}


%------------------------------------------------

\section{Solution 2/3: Hardware Virtualization (Intel VT)}
\begin{frame}
\frametitle{Solution 2/3: Hardware Virtualization (Intel VT)}
\begin{itemize}
\item \textbf{philosophy: instroduce new privileged mode}
\end{itemize}
\begin{figure}
\includegraphics[width=0.9\linewidth]{figures/vmx.pdf}
\end{figure}
\end{frame}

%------------------------------------------------

\section{KVM (Kernel-based Virtual Machine)}
\begin{frame}
\frametitle{KVM (Kernel-based Virtual Machine)}
\begin{columns}[c]
\column{.5\textwidth}
\begin{center}
\begin{itemize}
\item CPU hardware virtualization extensions (Intel VT or AMD-V)
\item Loadable kernel module (kvm.ko, kvm-intel.ko/kvm-amd.ko)
\item QEMU as userspace emulator
\end{itemize}
\end{center}
\column{.5\textwidth}
\begin{center}
\begin{figure}
\includegraphics[width=1.0\linewidth]{figures/kvm.pdf}
\end{figure}
\end{center}
\end{columns}
\end{frame}

%------------------------------------------------

\section{Solution 3/3: Paravirtualization}
\begin{frame}
\frametitle{Solution 3/3: Paravirtualization}
\begin{itemize}
\item \textbf{philosophy: replace critical instructions with hypercalls}
\item A hypercall is a software trap from a domain to the hypervisor, just as a syscall is a software trap from an application to the kernel
	\begin{itemize}
		\item x86\_32: \textit{int 0x82} 
		\item x86\_64: \textit{syscall instruction}
		\item x86 Intel-VT \textit{vmcall instruction}
	\end{itemize}
\end{itemize}
\begin{figure}
\includegraphics[width=1.0\linewidth]{figures/hypercall.pdf}
\end{figure}
\end{frame}

%Programs use syscall to enter the kernel, and the kernel is using syscall to
%enter xen. since you can only configure one entry point at a time (and both xen
%and applications run in ring 3 anyway), xen is handling all syscalls, then
%checks the context it was issued from, and forwards the call to to the guest
%kernel if it was user code.

%------------------------------------------------

\section{State of the Art Virtualization}
\begin{frame}
\frametitle{State of the Art Virtualization}
\begin{columns}[c]
\column{.68\textwidth}
\begin{itemize}
\visible<1->{\item Binary Translation (QEMU, Bochs, VMWare)}
\visible<2->{\item Paravirtualization (Xen)}
\visible<3->{\item Hardware-assisted Virtualization (KVM, Xen, VMware)}
\visible<4->{\item OS-level Virtualization (Linux Container)}
\visible<5->{\item Programming Language Virtualization (Java, .NET CLR)}
\visible<6->{\item Library Virtualization (Wine, Cygwin)}
\end{itemize}
\column{.32\textwidth}
\begin{figure}
\includegraphics[width=1.0\linewidth]{figures/virtlist.pdf}
\end{figure}
\end{columns}
\end{frame}

%------------------------------------------------

\section {What is Xen}
\begin{frame}
\frametitle{What is Xen}
\begin{block}{Wikipedia}
Xen Project is a hypervisor using a \textbf{\textcolor{red}{microkernel}} design, providing services that allow multiple computer operating systems to execute on the same computer hardware concurrently.
\end{block} \pause
\begin{block}{SOSP 2003: Xen and the Art of Virtualization}
This paper presents Xen, an x86 virtual machine monitor which allows multiple commodity operating systems to share conventional hardware in a safe and resource managed fashion, but without sacrificing either performance or functionality. \pause
\end{block}
\begin{block}{Basic Idea of Paravirtualization}
Actively inform the hypervisor with the action guest is going to taken via hypercall
\end{block}
\end{frame}

%------------------------------------------------

\section{Xen Framework 1/2}
\begin{frame}
\frametitle{Xen Framework 1/2}
\begin{block}{xen hypervisor (microkernel): dictator}
\begin{itemize}
\item scheduling, memory management, interrupt and device control
\item per-domain and per-vcpu info management
\end{itemize}
\end{block} \pause
\begin{block}{dom0 (host): privileged admin}
\begin{itemize}
\item xm/xend/xl (libxc)
\item pygrub/hvmloader
\item xenstored
\item qemu and paravirtual driver backend
\item native device driver
\end{itemize}
\end{block} \pause
\begin{block}{domU (guest): non-privileged user}
\begin{itemize}
\item paravirtual driver frontend
\end{itemize}
\end{block}
\end{frame}

%------------------------------------------------

\section{Xen Framework 2/2}
\begin{frame}
\frametitle{Xen Framework 2/2}
\begin{figure}
\includegraphics[width=1.0\linewidth]{figures/xen.pdf}
\end{figure}
\end{frame}

%------------------------------------------------

\section {Convert Linux to Paravirtual Dom0/DomU}
\begin{frame}
\frametitle{Convert Linux to Paravirtual Dom0/DomU}
\begin{itemize}
\item ELF notes (Linux) or \_\_xen\_guest section (MiniOS) in kernel image
\item Enable xen features in .config when building kernel
\end{itemize}
\begin{center}
\begin{figure}
\includegraphics[width=0.5\linewidth]{figures/elfnote.pdf}
\vspace{10 mm}
\includegraphics[width=0.4\linewidth]{figures/config.pdf}
\end{figure}
\end{center}

\end{frame}

%------------------------------------------------

\begin{frame}
\frametitle{PV, HVM or PVHVM}
\begin{figure}
\includegraphics[width=0.8\linewidth]{figures/spectrum.pdf}
\end{figure}
\end{frame}

%------------------------------------------------

\begin{frame}
\frametitle{Xen CPU Virtualization}
\begin{itemize}
\item vcpu $ \approx $ task\_struct
\item domain $ \approx $ container or process group
\item xen schedules vcpu
\end{itemize}
\begin{figure}
\includegraphics[width=1.0\linewidth]{figures/cpu.pdf}
\end{figure}
\end{frame}

%------------------------------------------------

\section{Xen Interrupt Virtualization: Event Channel 1/2}
\begin{frame}
\frametitle{Xen Interrupt Virtualization: Event Channel 1/2}
\begin{block}{Event Channel Types}
\begin{itemize}
\item Interdomain Event
\item Virtual IRQ Event
\item Physical IRQ Event
\item IPI Event
\end{itemize}
\end{block}
\begin{block}{Registration}
\begin{itemize}
\item PVM registers event channel handler to Xen via 

register\_callback(CALLBACKTYPE\_event, xen\_hypervisor\_callback)

\item PVHVM sets HYPERVISOR\_CALLBACK\_VECTOR via

HYPERVISOR\_hvm\_op(HVMOP\_set\_param, \&a)
\end{itemize}
\end{block}

\end{frame}

%------------------------------------------------

\section{Xen Interrupt Virtualization: Event Channel 2/2}
\begin{frame}
\frametitle{Xen Interrupt Virtualization: Event Channel 2/2}
\begin{figure}
\includegraphics[width=1.0\linewidth]{figures/evtchn.pdf}
\end{figure}
\end{frame}

%------------------------------------------------

\section{Xen Memory Virtualization 1/2}
\begin{frame}
\frametitle{Xen Memory Virtualization 1/2}
\begin{itemize}
\item Address Types
	\begin{itemize}
		\item GVA (Guest Virtual Address)
		\item GPA (Guest Physical Address) or GFN (Guest page Frame Number)
		\item HPA (Host Physical Address) or MFN (Machine page Frame Number)
	\end{itemize}
\item \textbf{\textcolor{red}{Hardware-assisted Memory Virtualization}} (Method 1/3): Second-Level Page Table
	\begin{itemize}
		\item: Intel: Extended Page Table (EPT)
		\item: AMD: Nested Page Table (NPT)
	\end{itemize}
\end{itemize}
\begin{figure}
\includegraphics[width=1.0\linewidth]{figures/ept.pdf}
\end{figure}
\end{frame}

%------------------------------------------------

\section{Xen Memory Virtualization 2/2}
\begin{frame}
\frametitle{Xen Memory Virtualization 2/2}
\begin{itemize}
\item \textbf{\textcolor{red}{Direct Paging}} (Method 2/3): guest manage the (GVA, HPA) page table directly
\item \textbf{\textcolor{red}{Shadow Paging}} (Method 3/3): xen hypervisor maintains a shadow (GVA, HPA) page table which is not awared by guest
\begin{figure}
\includegraphics[width=1.0\linewidth]{figures/shadow.pdf}
\end{figure}
\end{itemize}
\end{frame}

%------------------------------------------------

%\section{Xen Scheduling (credit2)}
%\begin{frame}
%\frametitle{Xen Scheduling (credit2)}
%\end{frame}

%------------------------------------------------

\begin{frame}
\frametitle{Xen Device Virtualization}
\begin{columns}[c]
\column{.6\textwidth}
\begin{itemize}
\visible<1->{\item HVM emulated legacy device (QEMU)}
\visible<2->{\item \textbf<5->{\textcolor<5->{red}{Paravirtual (PV) drivers}}}
\visible<3->{\item Device Passthrough (vt-d)}
\visible<4->{\item Virtual Function (vt-d)}
\end{itemize}
\column{.4\textwidth}
\begin{figure}
\includegraphics[width=1.0\linewidth]{figures/device.pdf}
\end{figure}
\end{columns}
\end{frame}

%------------------------------------------------

\section{PV driver vs. PCI driver}
\begin{frame}
\frametitle{PV driver vs. PCI driver}
\begin{table}
\begin{tabular}{l l l}
\toprule
& \textbf{PCI driver} & \textbf{PV driver}\\
\midrule
\textbf{device abstraction} & \textcolor{blue}{pci\_device, pci\_driver} & \visible<2->{\textcolor{red}{xenbus\_device, xenbus\_driver}} \\
\textbf{device discovery} & \textcolor{blue}{PCI Tree} & \visible<3->{\textcolor{red}{Xenstore}} \\
\textbf{device configuration} & \textcolor{blue}{PCI Config Space (IO/MMIO)} & \visible<4->{\textcolor{red}{Xenstore}} \\
\textbf{data flow} & \textcolor{blue}{DMA Ring Buffer} & \visible<5->{\textcolor{red}{Memory Ring Buffer}} \\
\textbf{shared memory} & \textcolor{blue}{N/A or IOMMU} & \visible<6->{\textcolor{red}{Grant Table}} \\
\textbf{interrupt} & \textcolor{blue}{IOAPIC, MSI, MSI-X} & \visible<7->{\textcolor{red}{Event Channel}} \\
\bottomrule
\end{tabular}
\end{table}
\begin{figure}
\includegraphics[width=.24\linewidth]{figures/fight.pdf}
\end{figure}
\end{frame}

%------------------------------------------------

\section{Xenstore/Xenbus}
\begin{frame}
\frametitle{Xenstore/Xenbus}
\begin{figure}
\includegraphics[width=1.0\linewidth]{figures/xenstore.pdf}
\end{figure}
\end{frame}

%------------------------------------------------

\section{Grant Table}
\begin{frame}
\frametitle{Grant Table}
\begin{figure}
\includegraphics[width=1.0\linewidth]{figures/grant.pdf}
\end{figure}
\end{frame}

%------------------------------------------------

\section{I/O Ring Buffer}
\begin{frame}
\frametitle{I/O Ring Buffer}
\begin{columns}[c]
\column{.6\textwidth}
\begin{center}
\begin{itemize}
\item Usually put grant ref (not data) in ring
\item Grant ref of ring pages are shared via xenstore
\item Usually one ring buffer for each device queue
\item One or more pages for each ring
\item Producer and Consumer (barrier)
\end{itemize}
\end{center}
\column{.4\textwidth}
\begin{center}
\begin{figure}
\includegraphics[width=1.0\linewidth]{figures/ring.pdf}
\end{figure}
\end{center}
\end{columns}
\end{frame}

%------------------------------------------------

\section{Xen Paravirtual Networking Framework}
\begin{frame}
\frametitle{Xen Paravirtual Networking Framework}
\begin{figure}
\includegraphics[width=0.6\linewidth]{figures/xennet.pdf}
\end{figure}
\end{frame}

%------------------------------------------------

\section{VM Creation Workflow}
\begin{frame}
\frametitle{VM Creation Workflow}
\begin{figure}
\includegraphics[width=1.0\linewidth]{figures/boot.pdf}
\end{figure}
\end{frame}

%------------------------------------------------

\section{Selected Xen Projects}
\begin{frame}
\frametitle{Selected Xen Projects}
\begin{columns}[c]
\column{.7\textwidth}
\begin{itemize}
\visible<1->{\item COLO - Coarse Grain Lock Stepping}
\visible<2->{\item LivePatch}
\visible<3->{\item Stealthy monitoring with Xen altp2m}
\visible<4->{\item Real-Time-Deferrable-Server(RTDS) CPU Scheduler}
\visible<5->{\item Windows PV Receive Side Scaling}
\visible<6->{\item More at Xen Summit and xen-devel}
\end{itemize}
\column{.3\textwidth}
\begin{center}
\begin{figure}
\includegraphics[width=.8\linewidth]{figures/projects.pdf}
\end{figure}
\end{center}
\end{columns}
\end{frame}

%------------------------------------------------

\section {Reference}
\begin{frame}
\frametitle{Reference}
\begin{block}{Publications}
\begin{itemize}
\item Xen and the art of virtualization. Paul Barham, Boris Dragovic, Keir Fraser, Steven Hand, Tim Harris, Alex Ho, Rolf Neugebauer, Ian Pratt, and Andrew Warfield. SOSP 2003
\item The Definitive Guide to the Xen Hypervisor. David Chisnall. 2007
\item Intel 64 and IA-32 Architectures Software Developer Manuals
\item Various system \& security research paper and presentation
\end{itemize}
\end{block}
\begin{block}{Miscellaneous}
\begin{itemize}
\item Xen Project Developer Summit
\item https://blog.xenproject.org
\item https://github.com/finallyjustice/JOS-vmx
\end{itemize}
\end{block}
\end{frame}

%------------------------------------------------

\section{Take-Home Message}
\begin{frame}
\frametitle{Take-Home Message}
\begin{columns}[c]
\column{.7\textwidth}
\begin{itemize}
\visible<1->{\item What is virtualization}
\visible<2->{\item Paravirtualization and Hardware-assisted Virtualization}
\visible<3->{\item Xen vs. KVM}
\visible<4->{\item Grant Table, Event Channel, Paravirtual Drivers}
\end{itemize}
\column{.3\textwidth}
\begin{center}
\begin{figure}
\includegraphics[width=.8\linewidth]{figures/home.pdf}
\end{figure}
\end{center}
\end{columns}
\end{frame}

%\begin{frame}
%\Huge{\centerline{The End}}
%\end{frame}

%----------------------------------------------------------------------------------------

\end{document} 
